%Lab report for first lab Liner Motion
\documentclass[12pt]{article}

\usepackage[utf8]{inputenc}
\usepackage[margin=1in]{geometry}

\title{Linear Motion}
\begin{document}

\begin{titlepage}
	\centering
    \topskip0pt
    \vspace*{\fill}
    {\scshape\huge Linear Motion \par}
    
    \vspace{2cm}
    \LARGE Nicholas Hall
    
    \large Lab Partners: Eric Battison and Brandon
    
    \vspace{2cm}
    \large Section 2
    \vspace*{\fill}
\end{titlepage}

\section{Introduction}

The main objective of this lab is to find the relationship between position, velocity, and acceleration over time for several motions of a nearly frictionless cart. In physics, the slope of a position over time graph corresponds to the velocity over time, and the slope of the velocity over time corresponds to acceleration. In this case, because the position over time graph is parabolic, or changed with an increasing or decreasing rate, the velocity changed linearly, and the acceleration was constant. The hypothesis is that when tracking the cart’s position, we will see this change and prove it with calculation. We will track the cart’s movement with a motion sensor and Data Studio to obtain our data. If the slopes of position over time, velocity over time, and acceleration over time correspond to each other our hypothesis will be confirmed.

\section{Procedure}
Equipment:
\begin{itemize}
	\item Cart, track, and bumpers
    \item Wood Block
    \item Motion Sensor
    \item Laptop with DataStudio and sensor interface
\end{itemize}
We will begin by observing the cart over three different situations: One with the cart on a completely flat track, one with the cart at the top of an incline, and one with the cart at the bottom of an incline. On a completely flat track we pushed the cart away from the sensor and hand charted a graph with a guess of its position, velocity, and acceleration over time. We repeated this step with the cart at the top of an incline, letting the cart fall away from the sensor. We repeated the observation stage a third time with the cart at the bottom of the incline, pushing up the track toward the sensor and letting the cart fall back to its initial position.

After observing and recording our guess we repeated the trials using the motion sensor and DataStudio to accurately record the position of the cart. A good trial was one that showed a smooth graph with few ``bumps''. 

\section{Data \& Analysis}

\end{document}





